% bitWeav whitepaper
% Written entirely by Liam (liamzebedee) Edwards-Playne.
% Licensed under Attribution-ShareAlike 3.0 Unported
% https://creativecommons.org/licenses/by-sa/3.0/

% Notes on formatting:
%  definitions of fields are bolded
%  reference to fields are emphasised
%  names of things are italicised


\documentclass[10pt,a4paper,onecolumn]{article}
\usepackage[utf8]{inputenc}
\usepackage{amsmath}
\usepackage{amsfonts}
\usepackage{amssymb}
\usepackage{url}
\usepackage{varioref}
\usepackage{enumerate}
\author{Liam Edwards-Playne \textless liamz.co\textgreater}
\date{June 1, 2013}
\title{bitWeav — scalable decentralised peer-to-peer communications}
\begin{document}
\maketitle
\begin{abstract}
\end{abstract}

\section{Introduction}
\label{intro}
In the last half-decade we have seen how the microblogging atmosphere of \textit{Twitter} has proven the value of online communication. The fundamental principle of microblogging, which is the compacting of ideas into 140 characters, has sparked a revolution in how we communicate. Users could spark a conversation on topics (tags) with strangers, discussing anything and everything in real-time. Through trending topics, this public yet often concealed discussion platform which is online communication could give rise to the sharing of influential ideas and the stirring of the masses. We could see this in events such as the Egyptian Revolution of 2011 and Occupy Wall Street. Such an invaluable resource must be kept sustainable, but we have seen that due to the centralised nature of this tool, as well as its issues with corporate control, there is a need for an alternative. 

A similar, older and more sustainable platform is \textit{IRC}. Like \textit{Twitter}, \textit{IRC} facilitated public text-based discussion on topics (channels). It had its advantages and disadvantages as it used a moderator-based group discussion device called a channel, but the user bases of such were fragmented across differing \textit{IRC} networks. Nonetheless, its more advanced organised discussion led to it becoming the key communications tool for 'hackers'. Hackers found the setup requiring minimum personal details, in combination with the simplistic and programmable interface which is the \textit{IRC} protocol, allowed for high-quality discussions that focused on the content. 

% TODO this is a bit too emotive
Finally there was \textit{Skype}. The unique feature of \textit{Skype} was its alternate method of communication — video chat. It allowed individuals to talk privately and securely with ease. Perhaps the most significant but relatively unknown aspect of \textit{Skype} was its decentralised and peer-to-peer network architecture. Although \textit{Skype} wasn't truly decentralised, it did show that service could be improved through such a system. Alas after the acquisition by \textit{Microsoft}, the platform headed downhill, instead revoking the previous security and decentralised nature of the application in favour of centralisation. Users could no longer rely on the simple yet secure platform which it had a strong image for, and have been lost in a myriad of alternatives since. 

One such alternative I have come across was \textit{Bitmessage}. \textit{Bitmessage} is a text-based communications protocol that rivals email, relying on a decentralised network of nodes to provide a peer-to-peer communications platform that is secure and easy to use. I commend the author of the application on these features and the effort it had taken to construct such a project, but despite these things the protocol had numerous faults. Its broadcast mechanism which resembled a gossip protocol was not to be desired. There had also been numerous reported vulnerabilities, but these extend the scope of my interest. I set out to build an alternative, that would combine the features of \textit{Twitter}, \textit{IRC} and \textit{Skype}, while retaining the sustainable nature of \textit{Bitmessage}. It was to be an experiment in decentralised peer-to-peer innovation.

\section{Objectives}
% don't talk about browser, talk about WebRTC html5 tech
Recent innovations and technologies have brought together a feasible environment for the creation of such a solution. The introduction of in-browser peer-to-peer networking API particularly struck me as a fantastic opportunity to implement an application that was decentralised and had magnificent ease-of-use. Another significant aspect was a novel decentralised peer-to-peer publish-subscribe architecture called \textit{PolderCast}, which featured a publish-subscribe system that distributed messages only to the nodes who were interested, in a scalable manner.

In this paper I describe both the theory of \textit{bitWeav} and its premier implementation called \textit{Weav}. To guide the development of \textit{bitWeav}, I formulated a list of objectives that expressed the key features necessary for its completion.
\begin{enumerate}
\item \textbf{Decentralised}: we must not rely on any set of nodes to provide for any other set of nodes, in the circumstances where if their connection were severed it would produce irreparable damage to the network. 

\item \textbf{Open}: we must rely totally on open technologies and likewise publish all concepts/implementations under a fair license.

\item \textbf{Simple}: the theory and implementation must be as simple as possible. the content distributed by the system must be contained within a single type — the message. All messages, contacts and topics must be uniquely identified by a single data type — the ID.

\item \textbf{Easy to use}: like Twitter and Skype the initial concept must be easy to understand for a user of these platforms, and likewise so must the user experience.

\item \textbf{Topic-enabled}: messages must be able to be categorised under multiple topics using tags, and distributed to those who are interested.

\item \textbf{Threaded}: threaded messaging must be inbuilt, wherein messages can be in reply to other messages, and descend from a common root (a thread)

\item \textbf{Privacy-enabled}: private distribution of messages must be supported, whereby only specific contacts will receive messages on a specific topic. These private conversations should pertain to the off-the-record messaging protocol. 

\item \textbf{Media-enabled}: contacts must be able to privately communicate through other mediums like video and audio. 

\item \textbf{Secure}: the authenticity, integrity, and completeness of public content sent within the network must be preserved by the system and independently verifiable by nodes. Replay attacks must be defended against. Forward secrecy must be ensured for private messages.
\end{enumerate}

\section{Contacts and messages}
\label{ContactsAndMessages}
At the core of \textit{bitWeav} are contacts and messages. Contacts are the users who access the \textit{bitWeav} network, and messages are the basis of their communications.

\subsection*{Contacts}
A contact is a generic term for machines and users accessing the \textit{bitWeav} network. Contacts authenticate using a \textbf{public/private key-pair}. They are identified by the computation of an identification function (described in ch.~\ref{id}) over the public key, which is referred to as their \textbf{address}. Since the public key is mathematically bound to the private key, this process makes forging an address infeasible.

The network operation of contacts is described in chapter~\ref{network}.

\subsection*{Messages}
A \textbf{message} is a generic container for micro-content. Micro-content is the basis of the micro-publishing idea: when content is small, it introduces constraints on users to relay more information into fewer words.

The bulk of the message is the \textbf{content}. The content is a maximum of 160 bytes in size and is interpreted as an array of \textit{UTF-8} characters. \textit{UTF-8} was chosen as the encoding, as it is a standardised format that has wide usage on the Web and has sufficient support for other alphabets. The 160 character limit (for Western alphabets) was chosen because it has shown sufficient for expressing ideas (text messages, \textit{Twitter} 'tweets') in threaded communications. 

The content can contain two types of markup: \textbf{hashtags} and \textbf{URIs}. Hashtags, which are space-delimited strings that begin with a hash '\#', are used to identify topics that a message is associated with. A message may have multiple hashtags. The other form of markup supported are links, which are \textit{URI}s which can be used to link to resources such as web pages and torrents (using the \textit{magnet URI} scheme).

The final part to the message is its authentication section, which contains two mandatory fields. The first is the \textbf{from} field, which is the public key of the publishing node. In conjunction with the second field the \textbf{signature}, which is a digital signature computed over the remainder of the message, this authenticates the author of all messages regardless of sender. Because of this, any node can forward a message to others, which is an essential feature of \textit{bitWeav}'s message distribution strategy.

These features alone provide something similar to \textit{Twitter} tweets and \textit{IRC} messages, but \textbf{lack any structure to the conversation}. Before I explain how threads are implemented, I must first describe how they are identified — the \emph{ID}.

\section{Identification}
\label{id}
As a requirement of the simple objective, we must have a single system of identification for contacts, messages and topics. I define a type called the \textbf{ID} to identify all of these items. 

The purpose of the ID is to provide a unique fixed-size identifier for a contact, message or hashtag. I make the distinction between \textbf{basic ID}s — those which are exchanged between computers, and \textbf{human ID}s — those which are exchanged between humans. The important difference between the two is that human IDs have additional steps for computation and encoding, including a recognisable version/identifier, a checksum to guard against input errors and Base58 encoding for ease of input. My human ID approach is the same algorithm for Bitcoin address generation\cite{btcAddr}. 

\subsubsection*{Basic ID generation}
\begin{enumerate}
\item Having some arbitrary array of data as input to the \emph{ID} function — the content.
\item Perform SHA2-256 hashing on the content.
\item Perform RIPEMD-160 hashing on the result of the SHA2-256 hashing.
\end{enumerate}

\subsubsection*{Human ID generation}
\begin{enumerate}
\item Given the result of basic ID generation from above as input to the \emph{humanID} function.
\item Prepend the version byte 0x49 or "W" to the basic ID.
\item Perform a double SHA2-256 hashing on the extended basic ID.
\item Append the first 4 bytes of the double SHA2-256 hashing (the checksum) to the extended basic ID.
\item Encode in Base58 for human representation. 
\end{enumerate}

For \textbf{contacts}, the ID is computed over the contact's public key. For \textbf{messages}, the ID is computed over the message's content (excluding the signature, which I describe in chapter~\ref{security}). Lastly for \textbf{topics} the ID is computed over the hashtag itself.

\section{Threading}
\subsection*{Reply-based threading}
There are two aspects to the threading of messages. The first is \textbf{reply-based threading} wherein a message is in response to another, which is what Twitter facilitates. The counterpart is \textbf{group-based threading}, wherein messages are part of a specific item of discussion on a topic — a thread. \textit{IRC} can be seen to support group-based threading, whereby each channel is a thread of messages. \textit{bitWeav} aims to support both of these. 

As each message can be identified using its ID, it is simply to implement reply-based threading. I define a field \textbf{reply}, which maps to the ID of the message that one is in reply to. There can only be one reply. 

\subsection*{Group-based threading}
The other type is group-based threading. In \textit{IRC} conversations, we have channels, which list a multitude of replies in a thread in chronological order. The model I described allows for a lossy form of discussion, but how would this channel concept be implemented? 

Firstly I define another field called the \textbf{root}, which maps to the ID of the original stimulus message for that conversation. When a node sends a message to a channel, they broadcast on the channel's topic. However this does not allow us to distinguish messages intended for a channel from normal microblogging 'noise'. Instead, we only listen for messages that are on a particular thread. 

This leaves us with only one problem, which is that of message ordering. The \emph{reply} field aids to develop a lossy ordering of messages. When we receive a message on the topic and thread we are listening on, we append it to the conversation thread according to which message it was in reply to. This allows for a lossy decentralised ordering of messages, but it is susceptible to nodes injecting messages into older parts of the chain. To guard against this, we employ a rolling metric to decide whether a message is recent enough: as the time differences between messages becomes smaller (and network synchronisation becomes harder), the accepted range for recent messages becomes larger. 

Due to the peer-to-peer nature of the overlay network, group messaging may be impeded by messages not being sent — unintentionally or on purpose. When we receive a message that is in reply to another message we have no record of, we contact the sending node to get the message. If the node does not respond, we discard their message. This method dissuades spammers from sending messages with falsified reply fields, as they will be flooded with queries. 

In this section I have outlined two optional fields: the reply and root fields. These  can be used to implement traditional threaded conversation, like that found in \textit{IRC} channels and \textit{Twitter} replies.

\section{Security}
\label{security}
A principle objective of \textit{bitWeav} is to be secure. The idea that security should be ubiquitous is fundamental to its design.

Nodes must communicate via a secure transport protocol to preserve the authenticity, integrity, and completeness of data sent within the network. This is further described in chapter~\ref{impl}.
% TODO elaborate on replay attack protections

\section{Overlay network}
\label{network}
The principle objective of the overlay network is to \textbf{distribute messages to interested nodes in a scalable and reliable manner}, a design called a publish-subscribe system. I considered a couple of publish-subscribe systems, namely \textit{SCRIBE}~\cite{castro2002scribe} and \textit{PUB-2-SUB}~\cite{pub2sub}, however I found their shared weakness to be relying on nodes which aren't necessarily interested in certain topics (so called 'rendez-vous' nodes) to aid in the distribution of messages on that topic.

I eventually discovered \textit{PolderCast}~\cite{poldercast}, a recent publish-subscribe system from 2012 that achieves this objective. The critical advantage of PolderCast lies in its gossip-based network design that is resilient to node churn while requiring only minimal node state necessary for routing — both publishers and subscribers are allowed to join and leave at any moment, without any prior notice, due to topic subscription being implicit in the routing process. It also features a layered network module design that proves useful when developing additional functions for the overlay network.

In this section I will outline the design of \textit{PolderCast} and describe my modifications to it.

\subsection{PolderCast}
The overlay network is composed of nodes and topics, both being uniquely identified by an ID (a basic ID as described in ch.~\ref{id}). \textbf{Nodes} are networked hosts that communicate with other nodes using a simple query-reply RPC framework (discussed in ch.~\ref{impl}). Nodes organise themselves around \textbf{topics}, to which they can subscribe to and publish messages on. At a conceptual level, each topic is modelled as a ring that connects all corresponding subscribers of it and only them. These rings are then augmented by random links across the topics which produces a single, connected and navigable overlay. 

The basis to the management of the overlay are three layered network modules: \emph{Cyclon}, \emph{Vicinity} and \emph{Rings}. Each module maintains its own routing table (called the view), managed by a separate gossiping protocol, which gossips periodically, asynchronously, and independently from the other two modules.

The \textbf{view} stores a list node profiles, each annotated respectively by an age value. A node~$Q$ being a \textbf{neighbour} of node~$P$ means that~$P$ has a copy of~$Q$'s profile in the respective module's view. A node's \textbf{profile} contains:
\begin{enumerate}[i.]
\item its network contact details.
\item its public key, and subsequently from this its unique ID or address (which is derived client-side).
\item its interests — a signed and timestamped copy of the IDs of topics the node is subscribed to, each annotated with a priority that node assigns to finding neighbours of that topic. 

Clearly the interests of a node may vary over time, so this section of a node's profile is timestamped so nodes may identify newer copies. It is also signed for obvious reasons of security.
\end{enumerate}

I also maintain two variables for each module: the view size~$V$ and gossip size~$G$. The \textbf{view size} defines the maximum number of neighbours to maintain, and is normally 20. The \textbf{gossip size} defines the maximum number of neighbours to be included in a gossip message, and is by default 20.

Finally, the network is based on a relatively small set of functions that provides for overlay network maintenance, message distribution and message retrieval, which I will proceed to describe.

\subsection{Overlay network maintenance}
As briefly mentioned, we have three modules, each with their own gossiping protocol. Each module's gossiping protocol is based on the concept of a \textbf{shuffle}, a process whereby nodes initiate neighbour exchanges periodically, yet not synchronized, with their oldest neighbour at a fixed period~$T$. I describe this shuffling process for each module for an initiating node~$P$. 

The common rules following gossiping are as follows:
\begin{enumerate}
\item Whenever a node gossips with another neighbour~$Q$, it temporarily removes~$Q$ from the respective module’s view, anticipating that it will respond and will be inserted anew. 
\item Whenever a node sends a message, it increments the age value of all nodes in the respective module's view.
\item A neighbour's age is retained when that neighbour is handed from one node to another.
\end{enumerate}

I define three RPCs in this category: \textbf{gossipCyclon}, \textbf{gossipVicinity} and \textbf{gossipRings}.

\subsubsection{Cyclon}
This module operates at the lowest-level, being situated just above the layer where communications take place. It forms an overlay and maintains random links between nodes using an epidemic algorithm. Each node knows a small, continuously changing set of neighbours, and occasionally contacts one to exchange some of their neighbours.

\paragraph*{Shuffling}
\begin{enumerate}
\item Select a neighbour $Q$ with the highest age among all neighbours, and a subset of $G-1$ other neighbours.
\item Replace $Q$'s entry with a new entry of age 0 and with $P$'s details.
\item Send the updated subset to $Q$.
\item Receive from $Q$ a subset of $G$ entries.
\item Discard entries pointing at $P$ and nodes that $P$ already knows of.
\item Verify entries are valid by establishing a connection to such nodes and discard any invalid entries.
\item Update $P$'s view to include all remaining entries, by firstly using empty slots (if any) and secondly replacing entries among the ones sent to $Q$.
\end{enumerate}

On reception of a shuffling query a node $Q$ replies by sending back a random subset of at most of $G$ its neighbours, and updates its own view to accommodate all received entries.

\paragraph*{Bootstrapping}
As \textit{Cyclon} is the lowest-level module, it is where we perform bootstrapping. To bootstrap the application, the bootstrap node is inserted into the \textit{Cyclon} module's view and three initial shuffles are performed with the node (to simulate a random walk, as per the original design).

\subsubsection{Vicinity}
This module maintains random links between nodes that share interest in one or more topics. Such links serve as input to the \textit{Rings} module and are used by the dissemination protocol to propagate events to arbitrary subscribers of a topic.

\paragraph*{Shuffling}
\begin{enumerate}
\item Let $P$ select its oldest neighbour $Q$ for gossiping.
\item $P$ merges its views from all three modules.
\item $P$ replaces $Q$'s entry with its own in this new merged view, setting the age to 0.
\item $P$ selects the $G$ nodes closest to $Q$ by applying the proximity function on its behalf, and sends them to $Q$. 

The proximity function is designed to ensure that the \textit{Rings} module is supplied with (arbitrary) neighbours for all its topics. In that respect, candidates subscribed to topics annotated with higher priority by the target node are ranked closer compared to candidates of lower priority topics. Among candidate nodes that rank equally in terms of topic priorities, proximity is determined by the number of topics shared with the target node: the more shared topics, the closer their ranking.
\item Receive from $Q$ its $G$ closest to $P$ nodes.
\item Merge these into the view and rebuild it by selecting the $G$ closest nodes and discarding the others.
\end{enumerate}

Upon reception of a shuffling query, $Q$ merges the received neighbours with the union of all its views, and updates its \textit{Vicinity} view to the $V$ closest neighbours. It then repeats steps 2 to 4 for $P$. 

\subsubsection{Rings}
This module discovers a node's successor and predecessor for each topic in its subscription, and quickly adapts to new successors/predecessors in dynamic networks.

Each node maintains $G$ neighbours for each topic in its subscription: $V/2$ with lower and $V/2$ with higher ID. It periodically picks a node from its \textit{Rings} view, and the two nodes exchange up to $G$ neighbours to help each other improve their respective views.

The sorting function used to determine the lower/higher nature of an ID is the XOR metric, as used in the Kademlia DHT.

\paragraph*{Shuffling}
\begin{enumerate}
\item Let $P$ select its oldest neighbour $Q$ for gossiping.
\item $P$ collects all subscribers of topics which $P$ and $Q$ have in common, considering the union of views of all three modules.
\item $P$ sorts them by ID, and for each topic in common with $Q$ it selects the $V/2$ ones with just lower and the $V/2$ ones with just higher ID than $Q$'s ID. 
\item If more than $G$ nodes have been selected, it randomly picks $G$ of them.
\item It sends the selected nodes to $Q$, and $Q$ does the same in return.
\item Upon reception of such nodes, $P$ merges them with the current view for the topic, and it selects the $V/2$ ones with just lower and the $V/2$ ones with just higher ID than $P$'s ID.
\end{enumerate}

% TODO message stuff (already written)
\subsection{Message distribution}
\subsection{Message retrieval}

% TODO private messaging, research OTR messaging
\section{Private communications}
% TODO protocol for video/audio chat and invites
\subsection*{Video/Audio communications}

% TODO spam prevention

\section{Implementation}
Nodes use a 512 bit elliptic curve keypair for their asymmetric encryption. Nodes use ECDSA with SHA-256 for the signature field of messages. Weav will automatically create a clickable link to the tag or the URI of a message.

% TODO overlay/IRC mechanism for finding nodes based on ID and signalling and crap

\section{Conclusion}
% This is what people read to get a sense of the innovations of the project, and is like the blurb of the paper.
Something here soon.

\bibliography{whitepaper}{}
\bibliographystyle{plain}

\end{document}
