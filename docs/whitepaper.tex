\documentclass[10pt,a4paper,onecolumn]{article}
\usepackage[utf8]{inputenc}
\usepackage{amsmath}
\usepackage{amsfonts}
\usepackage{amssymb}
\author{Liam Edwards-Playne \textless liamz.co\textgreater}
\title{BitWeav - a system for decentralised peer-to-peer autonomous communications}
\begin{document}
\maketitle
\begin{abstract}
\end{abstract}

\iffalse
archive messages on fellow nodes (those who are stored in address book)

kick - employ decentralised censorship of spam - if most nodes censor spam and don't forward it, the overall amount will go down.

structured private group messaging -- write out design for pub/sub, then get back to me
\fi

\section{Introduction}
At the heart of the Internet is communication. Twitter facilitates an disorganised form of discussion called \textit{micro blogging}, which pressures users to compact ideas into 140 characters, wherein they can be tagged and discussed in public among a large community of pseudonymous users. Skype supports more structured discussion threads, whereby any number of users can privately and securely discuss in a reply-based context. Finally there is IRC, which too has these reply-based group discussions, instead taking place on channels and being public. 

An aspect common to all three of these communications platforms is centralisation. Relying on a communications platform solely based on one's trust in the provider is a poor strategy. Decentralised platforms offer part of the solution, but their rigid structure of operations harms ease of use. What is needed is a hybrid between the two: a system which is decentralised in nature but employs trust to an extent, to improve overall function. 

We detail a decentralised peer-to-peer system for communications, that is implemented using web technologies. It features the most significant features of Skype, IRC and Twitter. It uses minimal trust features to enhance network function, but does not rely on it.

\section{Design}
The following objectives were formed to guide development:
\begin{enumerate}
\item \textbf{Decentralised}: we should not rely on any single entity to provide for any other entity, in the circumstances where if their connection were severed it would produce unrepairable damage. 
\item \textbf{Open}: we should rely totally on open technologies and likewise publish all BitWeav concepts/implementations under a fair license.
\item \textbf{Secure}: the authenticity, integrity, and completeness of content sent within the network must be preserved by the system and independently verifiable by nodes. Replay attacks must be defended against. Forward secrecy must be ensured for private messages.
\item \textbf{Simple}: the content distributed by the system should be contained within a single type - the message. All messages, contacts, topics should be uniquely identified by a single data type - the ID.
\item \textbf{Lightweight}: messages distributed by nodes should not exceed 256 bytes. 
\item \textbf{Topic-enabled}: messages should be able to be categorised under topics using tags.
\item \textbf{Threaded}: threaded messaging should be inbuilt, wherein messages can be in reply to other messages, and descend from a common root (thread)
\item \textbf{Privacy-enabled}: messages should be able to be marked as private, whereby only nodes who they are addressed to (if any) can read it
\item \textbf{Autonomous}: a user should simply require to input the details of a bootstrap node (IP address, contact address) to facilitate application function.
\end{enumerate}

While this paper describes a collection of concepts, it is implementation independent. However it must be noted that although the design was very important, the implementation of the autonomous setup objective was equally significant.


\section{Messages}
We define a message to be a generic container for micro-content. Micro-content is the basis of the micro-publishing idea: when content is small, it introduces constraints on users to relay more information into fewer words.  

The bulk of the message is the payload. The payload is 160 bytes in size, and is interpreted as an array of UTF-8 characters. UTF-8 was chosen as the encoding, as it is a standardised format that has wide usage on the Web and has sufficient support for other alphabets. The 160 character limit was chosen because it has shown sufficient for expressing ideas (text messages, Twitter tweets) in threaded communications. 

Within the payload, multiple tags can be defined that correspond to the topics that the message belongs to. Like Twitter tweets and IRC channels, tags are identified by a hash and delimited by spaces. The payload may contain multiple tags.


\section{Contacts}
I define a contact to be a generic term that encompasses machines and users accessing the BitWeav network. Contacts sign and perform other operations using their 384 bit elliptic curve public/private keypair. In comparison to AES, this keysize is what is recommended by the government for highly sensitive data [CITE].

For the overlay network, there is a necessity for unique keys amongst clients. The key or ID of the node is the identification function (described below) computed over the public key. Since the public key is mathematically bound to the private key, this process makes forging an address infeasible.


\section{Identification}
As a requirement of the 'simple' objective, we should have a single system of identification for contacts, messages and topics - the ID. I define the ID of these types to be the hash of their content. The ID is 224 bits in size and is computed using the SHA2-224 hash algorithm.

I employ SHA2-224 mainly because of its security and efficiency. As of April 2013, the feasibility of a hash collision is negligible. In contrast, the SHA1 family has had numerous significant attacks and is slower to compute. Finally, SHA3 is yet to be standardised. 


\section{Threading}
A message may optionally contain a 'reply' field, which stores the ID of the message it is in reply to. This field facilitates the reply mechanism of Twitter, but does not work well in group conversations (IRC, Skype) where message threads may span multiple periods of time, and thus would require a long-running identification token to listen for.

With this issue in mind, it is necessary to introduce the 'root' field, which identifies the original message that started the thread. The root field is  optional when the reply field is included (the reply is assumed to be the root), and ignored if no reply field exists.


\section{Privacy}
Private messaging, whether it be between individuals or groups, is achieved using a private field. Since we assume a secured transport layer (see \textit{Overlay network}), it is only necessary to include a field to distinguish between the visibility of messages. A node should not forward a message that has the private field set. 


\section{Overlay network}
The overlay network that this design assumes requires the following features:
\begin{itemize}
\item A bootstrap mechanism, to find nodes.
\item A publish mechanism, whereby messages are broadcast to interested nodes according to a topic (of type ID).
\item A subscribe mechanism, whereby a contact requests nodes to forward it messages on a topic.
\item A message retrieval mechanism, to get recent messages. 
\end{itemize}

We base our overlay network design on a topic-based P2P publish/subscribe architecture called PolderCast. Within Poldercast, nodes form rings according to topics that they subscribe to. 

We use an RPC framework to provide the following functions:
\begin{itemize}
\item PEER_SAMPLE:
\item  
\item PUBLISH
\item SUBSCRIBE
\item RECENT: to get recent messages distributed on a topic
\item SEARCH: to get particular messages and their subsequent replies
\end{itemize}



\section{Synchronisation}
Due to the peer-to-peer nature of the overlay network, group messaging may be impeded by messages not being sent - unintentionally or on purpose. When we receive a message that is in reply to another message we have no record of, we contact the sending node to get the message. If the node does not respond, we discard their message. This method dissuades spammers from sending messages with falsified reply fields, as they will be flooded with queries. 


\section{Profiles}
To aid in ease of use, we impose an address book and a subscription set. An address book stores pseudonym+contact pairs. A subscription set stores node identifiers. We assume that nodes who store eachother in their address books communicate frequently, and thus would benefit eachother by storing a backup of their information on eachothers 


\section{Implementation}
\subsection{Authentication}
One of the main issues with Retroshare and the like, is the need to transfer public/private keys. Cryptographic keypairs are not user-friendly, at least not for communications. In Bitcoin, users have adapted, but this is only because they are used less frequently than in social applications. What was needed was a user friendly motif of device-independent identification. 

In BitWeav I have taken an approach originally based on Bitcoin brainwallets. In BitWeav, every contact is authenticated by a public/private keypair, in comparison to social services whereby a username/password combination is used. As of current, we do not have a solution to providing the latter in a P2P manner. However we can provide a global identification mechanism by generating the private key using the hash of the username/password combination. 

There is an issue of nodes bruteforcing the private key. To guard against this, I employ firstly a password hashing algorithm (scrypt) to slow the process. The other thing I do is use the username in conjunction with the password, which guards against malicious nodes 'grabbing' many accounts from different users by simply bruteforcing common passwords. 

\subsection{Interface}
We host an open-source web application on weav.co. Users access the website through weav.co, which has a Cache Manifest to enable offline use. This allows users to access it by the familiar weav.co regardless of their Internet connection. 

We use the DataChannel aspect of WebRTC to facilitate the overlay network.  Bootstrapping is achieved by connecting to a list of known nodes.

The user interface is built using the Bootstrap framework and AngularJS. 

\subsection{Ease of use}
Web application. The web application splits messages into three views, with an additional filter field to show messages on a particular topic. These views are: disorganised, reply and thread. 

We employ third-party STUN servers to facilitate magnificently easy setup. 

\section{Future work}


Likewise, when this design is fully implemented, we intend to incorporate realtime audio/video call functionality, being the other significant aspect of WebRTC.

\end{document}
